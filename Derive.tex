\documentclass[12pt,a4paper]{article}
\usepackage[utf8]{inputenc}
\usepackage[russian]{babel}
\usepackage[OT1]{fontenc}
\usepackage{amsmath}
\usepackage{amsfonts}
\usepackage{amssymb}
\usepackage[left=2cm,right=2cm,top=2cm,bottom=2cm]{geometry}
\usepackage{ulem}
\author{Драгунъ Константинъ}
\title{Давайте производные изыматъ}
\date{27.09.1916}
\begin{document}
\maketitle

Ай-да, сударь, производную считать. Дело это, как широко известно, славное, богоугодное. Изымем производную какой-нибудь сложной функции.
Желательно подлиннее и пострашнее. Прямо чтобы \sout{мехматов} отроков ночью пугать можно было. 
Полагаю, эта сгодится:

\begin{center}
\begin{math}
\sin{\left({x}^{2}-3 \cdot x+5 \right )}
\end{math}
\end{center}
А теперь возьмем производную этой функции:

\begin{center}
\begin{math}
 \left ( \sin{\left({x}^{2}-3 \cdot x+5 \right )} \right)'=\cos{\left({x}^{2}-3 \cdot x+5 \right )}\cdot \left(\left({x}^{2}-3 \cdot x+5 \right ) \right)'
\end{math}
\end{center}
А теперь возьмем производную этой функции:

\begin{center}
\begin{math}
 \left ( \left({x}^{2}-3 \cdot x+5 \right ) \right)'= \left ( {x}^{2}-3 \cdot x \right)'+ \left ( 5 \right)'
\end{math}
\end{center}
А теперь возьмем производную этой функции:

\begin{center}
\begin{math}
 \left ( {x}^{2}-3 \cdot x \right)'= \left ( {x}^{2} \right)'- \left ( 3 \cdot x \right)'
\end{math}
\end{center}
А теперь возьмем производную этой функции:

\begin{center}
\begin{math}
 \left ( 3 \cdot x \right)'= \left ( 3 \right)' \cdot x+3\cdot \left ( x \right)'
\end{math}
\end{center}
А теперь возьмем производную этой функции:

\begin{center}
\begin{math}
 \left ( {x}^{2} \right)'=2 \cdot \left(x \right)^{2- 1}
\end{math}
\end{center}
So, derive is:

\begin{center}
\begin{math}
 \left ( {x}^{2}-3 \cdot x \right)'={x}^{1} \cdot 2 \cdot 1-0 \cdot x+3 \cdot 1
\end{math}
\end{center}
So, derive is:

\begin{center}
\begin{math}
 \left ( \left({x}^{2}-3 \cdot x+5 \right ) \right)'={x}^{1} \cdot 2 \cdot 1-0 \cdot x+3 \cdot 1+0
\end{math}
\end{center}
Очевиднейшим образом находим ответ.

\begin{center}
\begin{math}
 \left ( \sin{\left({x}^{2}-3 \cdot x+5 \right )} \right)'=\cos{\left({x}^{2}-3 \cdot x+5 \right )} \cdot \left(x \cdot 2-3 \right )
\end{math}
\end{center}
\end{document}
